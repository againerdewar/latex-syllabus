\documentclass[11pt,article]{memoir}
\usepackage{agd-syllabus}

\instructorname{Dr.\ Andrew Gainer-Dewar}
\instructoremail{againerdewar@carleton.edu}
\instructoroffice{}
\instructorofficehours{}
\instructorphone{}

\coursename{}
\coursenum{}
\courseinst{}
\courseterm{}
\coursetime{}
\courseroom{}

\begin{document}
\maketitle

\section*{Course Description}

\section*{Text}

\section*{Grades}

\section*{Homework}
Homework is an important part of your learning process, and you should take it seriously.
Some problems I assign are intended to demonstrate new concepts; other problems are intended to drill skills you already know.
Both are crucial to the process of learning the content of the course.

\textbf{I will not accept homework late for any reason except as required by the College.}
I will give you at least one week's notice of the due date of any assignment.

Please be respectful of the grader's time and attention.
Write your homework neatly or typeset it with a computer.
Staple pages together to ensure that none are lost.

\section*{Use of technology}
Calculators and other technology are \emph{not} allowed on the exams.
Accordingly, you should not become dependent on them.
Technological tools can be valuable for checking your answers on homework problems, but you must know how to do all the work yourself, and you must show it on the homework.

\section*{Collaboration}
Collaboration with your fellow mathematicians is an important skill.
Working together can help everyone in a group to get over creative and technical hurdles and lead to increased understanding.
Moreover, it provides valuable experience in communicating effectively.

I therefore encourage you to work on your homework in small groups.
However, it is essential that your work reflect your own understanding.
To demonstrate this, \emph{you must submit work that is entirely your own, with no text copied from or written together with another student}.
Please also consult the Carleton College handbook section on Academic Honesty.
If you have any questions about this policy or how it applies to a particular situation, please see me about them \emph{before} submitting the work in question.

\end{document}